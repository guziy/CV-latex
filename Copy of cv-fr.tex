% LaTeX Curriculum Vitae Template
%
% Copyright (C) 2004-2009 Jason Blevins <jrblevin@sdf.lonestar.org>
% http://jblevins.org/projects/cv-template/
%
% You may use use this document as a template to create your own CV
% and you may redistribute the source code freely. No attribution is
% required in any resulting documents. I do ask that you please leave
% this notice and the above URL in the source code if you choose to
% redistributemoti this file.

\documentclass[letterpaper]{article}

\usepackage{hyperref}
\usepackage{geometry}
\usepackage[utf8x]{inputenc}
\usepackage{datetime}

% Comment the following lines to use the default Computer Modern font
% instead of the Palatino font provided by the mathpazo package.
% Remove the 'osf' bit if you don't like the old style figures.
\usepackage[T1]{fontenc}
\usepackage[sc]{mathpazo}
\usepackage{longtable}


% Set your name here
\def\name{Oleksandr Huziy}

% Replace this with a link to your CV if you like, or set it empty
% (as in \def\footerlink{}) to remove the link in the footer:
\def\footerlink{}

% The following metadata will show up in the PDF properties
\hypersetup{
  colorlinks = true,
  urlcolor = black,
  pdfauthor = {\name},
  pdfkeywords = {economics, statistics, mathematics},
  pdftitle = {\name: Curriculum Vitae},
  pdfsubject = {Curriculum Vitae},
  pdfpagemode = UseNone
}

\geometry{
  body={6.5in, 8.5in},
  left=1.0in,
  top=1.25in
}

% Customize page headers
\pagestyle{myheadings}
\markright{\name}
\thispagestyle{empty}

% Custom section fonts
\usepackage{sectsty}
\sectionfont{\rmfamily\mdseries\Large}
\subsectionfont{\rmfamily\mdseries\itshape\large}

% Other possible font commands include:
% \ttfamily for teletype,
% \sffamily for sans serif,
% \bfseries for bold,
% \scshape for small caps,
% \normalsize, \large, \Large, \LARGE sizes.

% Don't indent paragraphs.
\setlength\parindent{0em}

% Make lists without bullets
\renewenvironment{itemize}{
  \begin{list}{}{
    \setlength{\leftmargin}{1.5em}
  }
}{
  \end{list}
}

\begin{document}

% Place name at left
{\huge \name}

% Alternatively, print name centered and bold:
%\centerline{\huge \bf \name}

\vspace{0.25in}

\begin{minipage}{0.45\linewidth}
  Email: \href{mailto:guziy.sasha@gmail.com}{\tt guziy.sasha@gmail.com}
\end{minipage}




\section*{Éducation}

\setlength\LTleft{0pt}
\setlength\LTright{0pt}

\begin{longtable}{@{}l@{\extracolsep{\fill}}r@{}}
\begin{minipage}[t]{10cm}
\textbf{Doctorat en sciences de la Terre et de l'atmosphère} \newline 
\emph{UQÀM (Université du Québec à Montréal)}
\end{minipage}
&
05/2010 - présent \\ \\

\begin{minipage}[t]{10cm}
\textbf{Maîtrise en sciences de l'atmosphère (les cours seulement)}\newline 
\emph{UQÀM (Université du Québec à Montréal)}
\end{minipage}
&
09/2009 - 04/2010 \\ \\

\begin{minipage}[t]{10cm}
\textbf{Maîtrise en physique et mathématiques appliquées, département de
contrôle/gestion et de mathématiques appliquées}\newline 
\emph{MIPT (Institut de Moscou de physique et de la technologie, département à Kiev), département de contrôle/gestion et
de mathématiques appliquées.} \newline
Projet final: Développement du systéme de prévision du temps pour Ukraine.
\end{minipage}
&
09/2006 - 06/2008 \\ \\

\begin{minipage}[t]{10cm}
\textbf{Baccalauréat en physique et mathématiques appliquées,
département des Problèmes de physique et d'énergétique}\newline 
\emph{MIPT (Institut de Moscou de physique et de la technologie), département
des problèmes de physique et génie energetique.} \newline
Projet final: Modélisation du transport des substances radioactives dans
l'atmosphère.
\end{minipage}
&
09/2002 - 06/2006 \\ \\
\end{longtable}


\section*{Expérience}
\begin{itemize}
  \item (10/2012-12/2012)
  \textbf{auxiliare d'enseignement (démonstrateur)}, cours: Hydrologie,
  Université du Québec à Montréal.\\
  Fonctions:
  \begin{enumerate}
    \item Animation et correction des travaux pratiques.
    \item Support technique.
  \end{enumerate}
   
  \item (02/2012 - 03/2012, 05/2012):
  \textbf{assistant de recherche} centre ESCER (l'étude et simulation du
  climat à l'échelle régionale), Université du Québec à Montréal.\\
  Fonctions:
  \begin{enumerate}
    \item Analyse des sorties des modèles régionaux du climat.
    \item Support avec applications SIG.
  \end{enumerate}


  \item (08/2008 - 10/2008): 
   \textbf{chercheur junior} à l'Institut de problèmes des machines et des
   systèmes de mathématique, Kiev, Ukraine. \\
  Fonctions:
  \begin{enumerate}
    \item Support scientifique du sustème de prévision du temps
    \href{http://meteoprog.com}{http://meteoprog.com}.
    \item Développement du système de support des décisions pour la
    gestion des situations d'urgence nucléaire en Europe (RODOS).
  \end{enumerate}
\end{itemize}
  


\section*{Réalisations}

\begin{itemize}
   
    \item Jacinthe Clavet-Gaumont, Laxmi Sushama, Naveed Khaliq,
    \textbf{Oleksandr Huziy}, René Roy, "Canadian RCM projected changes to high
    flows for Québec watersheds using regional frequency analysis", International Journal of Climatology.
    
    \item \textbf{Oleksandr Huziy}, Laxmi Sushama, Naveed Khaliq, Bernhard
    Lehner, René Laprise, René Roy, "Analysis of Streamflow Characteristics over Northeastern Canada 
     in a Changing Climate", Climate Dynamics, 2012, DOI:
     10.1007/s00382-012-1406-0.
    \item \textbf{Oleksandr Huziy}, Laxmi Sushama, Naveed Khaliq, Bernhard Lehner,
     René Laprise, René Roy, "Analysis of Streamflow Characteristics over Northeastern Canada 
     in a Changing Climate", poster presentation at CMOS conference, Montreal,
     28 May - 1 June 2012.
    \item \textbf{Oleksandr Huziy}, Laxmi Sushama, Naveed Khaliq, Bernhard Lehner, René
    Laprise, René Roy: "Projected Changes to Streamflow Characteristics in
    North-East Canadian Basins as Simulated by the Canadian Regional Climate
    Model (CRCM4)", poster presentation at AGU Fall meeting, San Francisco,
    December 5, 2011.
    \item \textbf{Oleksandr (Sasha) Huziy}, Jacinthe Clavet-Gaumont, Laxmi
    Sushama, Naveed Khaliq, Bernhard Lehner, René Laprise, René Roy: "Projected changes to streamflow extremes in Quebec
    basins as simulated by the Canadian Regional Climate Model (CRCM4)",
    presentation at workshop at ESCER, UQÀM, Montreal (QC), May 25, 2011,
    (\url{http://people.sca.uqam.ca/~huziy/pdf/Workshop_UQAM_2011.pdf}).
   
    \item \textbf{Oleksandr Huziy}, Devis de recherche, "Development of a
    Lake-River Module for Canadian Regional Climate Model (CRCM5)", 2011,
    (\url{http://people.sca.uqam.ca/~huziy/pdf/devis_Huziy.pdf}).
   
    \item \textbf{Oleksandr Huziy}, "Hydrological Cycle in Current and Future Climate",
    projet réalisé dans le cadre du cours STA-9800: Dynamique du globe, 2011,\\
     (\url{http://people.sca.uqam.ca/~huziy/pdf/GlobeDynamicsReport.pdf}).
   
    \item \textbf{Oleksandr Huziy}, "Development of flow directions for use in
    distributed river models", projet réalisé dans le cadre du cours STA9850
    Concept de système en sciences de la Terre et de l'atmosphère, 2011,
    (\url{http://people.sca.uqam.ca/~huziy/pdf/flow_directions.pdf}).
   
    \item \textbf{Oleksandr Huziy}, "Atmospheric Boundary Layer over the Lake Surface",
           projet réalisé dans le cadre du cours "Couche limite atmosphérique",
           2010, (\url{http://people.sca.uqam.ca/~huziy/pdf/Report_Huziy_ABL.pdf}).
   
    \item \textbf{Oleksandr Huziy}, "Modélisation d’un nuage chaud stratiforme",
       projet réalisé dans le cadre du cours "Convection et precipitation", 2010,
       (\url{http://people.sca.uqam.ca/~huziy/pdf/Huziy_SCA7050_rapport.pdf}).

    \item Oleksandr Huziy et André Monette, "Identification et Prévision du
    Brouillard en Utilisant les Données Satellitaires", projet réalisé dans le cadre du cours Météorologie et télédetection,
       2010, (\url{http://people.sca.uqam.ca/~huziy/pdf/Report_Monette_HUziy_teledetection.pdf}).

    \item \textbf{Guziy A.M. (Huziy O. M.)}, Kovalets I.V., Kushchan A.A.,
    Zheleznyak M.J.
    "Statistical downscaling of results of the forecasting system MM5–Ukraine using neural net technology.",
    workshop presentation, Saint-Petersburg, 2008, (\url{http://people.sca.uqam.ca/~huziy/pdf/presentation_spb2008.pdf}).

    \item I. Ievdin, D. Treebushny, I. Kovalets, \textbf{O. Guziy(Huziy O. M.)}, M.I.
    Zheleznyak, "Development Multi-Platform Version Of Decision Support System For Nuclear Emergency Management In Europe (Rodos)
    On Base Of Modern Java And Gis Technologies", 
    Institute of Mathematical Machine and System Problems NAS of Ukraine, conference
    "Decision Support Systems. Theory and Practice", pp.121 - 124, Kiev, 2008.
     (\url{http://people.sca.uqam.ca/~huziy/pdf/ievdin_sppr2008.pdf})

\end{itemize}


\section*{Bourses et distinctions}

\begin{itemize}
    \item (2009-2012, trimestre d'hiver 2013): Bourses d'exemptions des droits
    de scolarité supplémentaires pour les étudiants étrangers données par le
    gouvernement du Québec.
    \item (2012): Bourse d'éxcellence d'Hydro-Québec. 
    \item (2011): Bourse d'éxcellence d'Hydro-Québec. 
    \item (2010): Bourse d'éxcellence d'Hydro-Québec.
    \item (2009): Bourse d'éxcellence FARE (fonds à l'accessibilité et à la
    réussite des études).
    
\end{itemize}




\bigskip

% Footer
\begin{center}
  \begin{footnotesize}
    Dernier mis à jour: \ddmmyyyydate \today \\
    \href{\footerlink}{\texttt{\footerlink}}
  \end{footnotesize}
\end{center}

\end{document}

