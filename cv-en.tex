% LaTeX Curriculum Vitae Template
%
% Copyright (C) 2004-2009 Jason Blevins <jrblevin@sdf.lonestar.org>
% http://jblevins.org/projects/cv-template/
%
% You may use use this document as a template to create your own CV
% and you may redistribute the source code freely. No attribution is
% required in any resulting documents. I do ask that you please leave
% this notice and the above URL in the source code if you choose to
% redistributemoti this file.

\documentclass[12pt,a4paper,sans]{moderncv}

\usepackage{datetime}


%----------------------------------------------------------------------------------------
%   NAME AND CONTACT INFORMATION SECTION
%----------------------------------------------------------------------------------------

\firstname{Oleksandr} % Your first name
\familyname{Huziy} % Your last name

% All information in this block is optional, comment out any lines you don't need
\title{Curriculum Vitae}
\email{guziy.sasha@gmail.com}
\homepage{www.escer.uqam.ca/huziy_EN.html}{www.escer.uqam.ca/huziy\_EN.html}
\extrainfo{My blog: guziy.blogspot.com}
%\photo[70pt][0.4pt]{pictures/Huziy}


\moderncvstyle{classic} % CV theme - options include: 'casual' (default),
% 'classic', 'oldstyle' and 'banking'
\moderncvcolor{blue} % CV color - options include: 'blue' (default), 'orange',
% 'green', 'red', 'purple', 'grey' and 'black'



\usepackage[scale=0.8]{geometry} % Reduce document margins
\setlength{\hintscolumnwidth}{4cm} % Uncomment to change the width of the dates column
%\setlength{\makecvtitlenamewidth}{10cm} % For the 'classic' style, uncomment to adjust the width of the space allocated to your name


\usepackage[utf8x]{inputenc}


\newcommand{\vertspace}{1em}


\begin{document}


\makecvtitle

% Set the spacing between paragraphs and items
\setlength{\parskip}{\vertspace}



\section{Education}
\cventry{05/2010--03/2016}{Ph.D. in Earth and Atmospheric Sciences}{}{}{}{UQÀM
(Université du Québec à Montréal)\newline Thesis: "Study of lake-river-atmosphere interactions for northeast Canada in current and future climates"}

\cventry{09/2009--04/2010}{M.Sc. in Atmospheric Sciences (classes only)}{}{}{}{UQÀM
(Université du Québec à Montréal)}

\cventry{09/2006--06/2008}{M.Sc. in applied physics and mathematics}{}{}{}{MIPT
(Moscow Institute of Physics and Technology, Kiev branch), department of
Control/Management and Applied Mathematics.\newline
Final dissertation: Development of weather prediction system for
Ukraine.}


\cventry{09/2002--06/2006}{B.Sc. in applied physics and mathematics}{}{}{}{MIPT
(Moscow Institute of Physics and Technology), department of
Problems of Physics and Power Engineering. \newline Final dissertation: Modeling of transport of radioactive substances in the
atmosphere.}


\section{Experience}
\vspace{\vertspace} % You can remove this when the entries are converted to cventry

% \cventry{years}{degree/job title}{institution/employer}{localization}{grade}{description}

%TODO: Add TA, RA and CNRCWP coordination

% PDF at UQAM
\cventry{04/2016--present}{Postdoctoral Fellow}{centre ESCER, Université du Québec à Montréal}{Montreal, Canada}{}{
Responsibilities:
\begin{itemize}
\setlength\itemsep{0em}
  \item parameterization of large lakes in the fifth generation of Canadian Regional Climate Model (CRCM5) using a 3D ocean model NEMO.
  \item Study lake-effect snowfall events in current and future climates
\end{itemize}
}



%%%%%%%%%%%%%%%%%%%%%%%% RA at UQAM
\cventry{09/2013--01/2016}{Research assistant}{centre ESCER, Université du Québec à Montréal}{Montreal, Canada}{}{
Responsibilities:
\begin{itemize}
\setlength\itemsep{0em}
  \item parameterization of large lakes in the fifth generation of Canadian Regional Climate Model (CRCM5)
\end{itemize}
}


%%%%%%%%%%%%%%%%%%%%%%%%% TA at UQAM
\cventry{9/2013---12/2015}{Teaching assistant}{Université du Québec à Montréal, course: Hydrology}{Montreal, Canada}{}{
Responsibilities:
\begin{itemize}\setlength\itemsep{0em}
  \item Animation of labs and correction of reports.
  \item Technical support
\end{itemize}
}




%%%%%%%%%%%%%%%%%%%%%%%% CNRCWP coordinator
\cventry{07/2013--01/2016}{Coordinator}{Canadian Network for Regional Climate and Weather Processes}{Montreal, Canada}{}{
Responsibilities:
\begin{itemize}
\setlength\itemsep{0em}
\item Maintenance and development of the Network website \url{cnrcwp.uqam.ca}.
\item Organizing meetings and facilitating communication.
\end{itemize}
}



\cventry{02/2012--03/2012, 05/2012}{Research assistant}{ESCER, Université du Québec à Montréal}{Montreal, Canada}{}{
  Responsibilities:
  \begin{itemize}
  \setlength\itemsep{0em}
  \item Analysis of output from regional climate models.
  \item Support with GIS applications.
  \end{itemize}}

\cventry{08/2008--10/2008}{Junior researcher}{Institute of Mathematical
  Machines and Problems}{Kiev, Ukraine}{}{
  Responsibilities:
  \begin{itemize}
  \setlength\itemsep{0em}
    \item Scientific support of a weather forecasting system at
    \href{http://meteoprog.com}{http://meteoprog.com}.
    \item Development of a decision support system for nuclear emergency
    management in Europe (Rodos).
  \end{itemize}}

\section{Relevant Skills}
\vspace{\vertspace}
\begin{itemize}

  \item Used parallel models WRF (Weather Research and Forecasting) and CRCM
  (Canadian Regional Climate Model), and wrote parallel programs in Fortran
  (MPI and OpenMP), Python (multithreading and multiprocessing) and Java (threads).

  \item Used for analysis the following models in standalone mode: CLASS (Canadian Land Surface Scheme), Hostetler (1D lake model) and NEMO (Nucleus for European Modelling of the Ocean).

  \item Analysis of large datasets (model simulations and observation datasets) using Python.

  \item Developed a python wrapper for rmnlib (FORTRAN/C library for reading and writing standard RPN files) using ctypes module. The wrapper is available on github at \href{https://github.com/guziy/pylibrmn}{https://github.com/guziy/pylibrmn}.

  \item Developed programs and solved problems using the following programming languages: Python, Java, MATLAB, C/C++, Fortran, R, Linux shell scripting, JavaScript.

  \item Comfortable with git and svn version control systems.

  \item Enthusiastic about open-source Python modules (Basemap, Matplotlib,
  SciPy, Pandas and others) and try to contribute to the development (by
  submitting issues and pull requests) and to the popularization (by answering
  questions on mailing lists or helping with issues submitted to
  issue tracking systems).

  \item Intermediate English and French language skills. Writing and oral communication skills are not perfect.

\end{itemize}


\section*{Peer reviewed publications}
\vspace{\vertspace}


{
\renewcommand{\labelitemi}{}
\begin{itemize}

    \item \textbf{Huziy O.} and Sushama L., 2015: "Lake-river and lake-atmosphere interactions in a changing climate over Northeast Canada". Submitted

    \item \textbf{Huziy O.} and Sushama L., 2015: "Impact of lake-river connectivity and interflow on the Canadian RCM simulated regional climate and hydrology for Northeast Canada". Submitted

    \item Clavet-Gaumont J., Sushama L., Khaliq M.N.,
    \textbf{Huziy O.}, Roy R., 2013: "Canadian RCM projected changes to high
    flows for Québec watersheds using regional frequency analysis", International Journal of Climatology.

    \item \textbf{Huziy O.}, Sushama L., Khaliq M.N., Bernhard
    Lehner, René Laprise, Roy R., 2013: "Analysis of Streamflow Characteristics over Northeastern Canada
     in a Changing Climate", Climate Dynamics, DOI:
     10.1007/s00382-012-1406-0.

\end{itemize}
}



\section*{Conferences and seminars}

\vspace{\vertspace}

{
\renewcommand{\labelitemi}{}
\begin{itemize}


  \item \textbf{Huziy O.}, Sushama L., 2016. Lake-river and lake-atmosphere
  interactions in a changing climate over Northeast Canada, CMOS, Fredericton, Canada.

  \item \textbf{Huziy O.}, Sushama L., 2016. Lake-river and lake-atmosphere
  interactions in a changing climate over Northeast Canada, EGU, Vienna, Austria.

   \item Duguay C., Sushama L., \textbf{Huziy O.}, Baijnath J., Music B., 2015.
   Representation of large lakes in weather and climate models. CNRCWP Science
   meeting, Montreal.

   \item \textbf{Huziy O.}, Sushama L., 2015. Exploring interflow - soil moisture - climate
   linkages: Case study for northeast Canada. Joint assembly AGU-GAC-MAC-CGU, Montreal.

   \item \textbf{Huziy O.}, Sushama L., Laprise R., 2014. Lake-river-atmosphere Interactions
   as Simulated by the Canadian Regional Climate Model (CRCM5) over North-east
   Canada. AGU Fall meeting, San Francisco.

   \item \textbf{Huziy O.}, Sushama L., Laprise R., 2014. Lakes and rivers in
   CRCM5 and 3D simulations of The Great Lakes. CNRCWP Science meeting, Montreal

   \item \textbf{Huziy O.}, Sushama L., Khaliq M.N., Laprise R., Lehner B. and Roy
    R., 2013. Analysis of streamflow characteristics over northeastern Canada in a
    changing climate. Canadian Meteorological and Oceanographic Society (CMOS),
    Saskatoon.

   \item \textbf{Huziy O.}, Sushama L. and Laprise R., 2013. Impact of lakes
   and rivers on the regional climate and hydrology of northeast Canada.
   Workshop on Regional Climate modelling and Diagnostics, UQAM, Montreal.

    \item \textbf{Huziy O.}, Sushama L., Khaliq M.N., Lehner B., Laprise R., Roy
    R., "Analysis of Streamflow Characteristics over Northeastern Canada in a
    Changing Climate", poster presentation at CMOS conference, Montreal, 28 May -
    1 June 2012.

    \item \textbf{Huziy O.}, Sushama L., Khaliq M.N., Lehner B., Laprise R., Roy
    R.: "Projected Changes to Streamflow Characteristics in North-East Canadian
    Basins as Simulated by the Canadian Regional Climate Model (CRCM4)", poster
    presentation at AGU Fall meeting, San Francisco, December 5, 2011.

    \item \textbf{Huziy O.}, Clavet-Gaumont J., Sushama L., Khaliq M.N., Lehner
    B., Laprise R., Roy R.: "Projected changes to streamflow extremes in Quebec
    basins as simulated by the Canadian Regional Climate Model (CRCM4)",
    presentation at workshop at ESCER, UQÀM, Montreal (QC), May 25, 2011,
    (\url{http://people.sca.uqam.ca/~huziy/pdf/Workshop_UQAM_2011.pdf}).

    \item \textbf{Guziy A.M. (Huziy O. M.)}, Kovalets I.V., Kushchan A.A.,
    Zheleznyak M.J.
    "Statistical downscaling of results of the forecasting system MM5–Ukraine using neural net technology.",
    workshop presentation, Saint-Petersburg, 2008, (\url{http://people.sca.uqam.ca/~huziy/pdf/presentation_spb2008.pdf}).

\end{itemize}
}



\section{Scholarships}
\vspace{\vertspace}
    \cvitem{2009--2012}{Fee waiver scholarship for international students given
    by government of Quebec.}
    \cvitem{2011}{Excellence scholarship from Hydro-Quebec}
    \cvitem{2010}{Excellence scholarship from Hydro-Quebec}
    \cvitem{2009}{Excellence scholarship FARE}


\section*{Reports}
\vspace{\vertspace}
\begin{itemize}
    \item \textbf{Huziy O.}, Devis de recherche, "Development of a
    Lake-River Module for Canadian Regional Climate Model (CRCM5)", 2011,
    (\url{http://people.sca.uqam.ca/~huziy/pdf/devis_Huziy.pdf}).

    \item \textbf{Huziy O.}, "Hydrological Cycle in Current and Future Climate",
    projet réalisé dans le cadre du cours STA-9800: Dynamique du globe, 2011,\\
     (\url{http://people.sca.uqam.ca/~huziy/pdf/GlobeDynamicsReport.pdf}).

    \item \textbf{Huziy O.}, "Development of flow directions for use in
    distributed river models", projet réalisé dans le cadre du cours STA9850
    Concept de système en sciences de la Terre et de l'atmosphère, 2011,
    (\url{http://people.sca.uqam.ca/~huziy/pdf/flow_directions.pdf}).

    \item \textbf{Huziy O.}, "Atmospheric Boundary Layer over the Lake Surface",
           projet réalisé dans le cadre du cours "Couche limite atmosphérique",
           2010, (\url{http://people.sca.uqam.ca/~huziy/pdf/Report_Huziy_ABL.pdf}).

    \item \textbf{Huziy O.}, "Modélisation d’un nuage chaud stratiforme",
       projet réalisé dans le cadre du cours "Convection et precipitation", 2010,
       (\url{http://people.sca.uqam.ca/~huziy/pdf/Huziy_SCA7050_rapport.pdf}).

    \item \textbf{Huziy O.} et Monette A., "Identification et Prévision du
    Brouillard en Utilisant les Données Satellitaires", projet réalisé dans le cadre du cours Météorologie et télédetection,
       2010, (\url{http://people.sca.uqam.ca/~huziy/pdf/Report_Monette_HUziy_teledetection.pdf}).


    \item Ievdin I., Treebushny D., Kovalets I., \textbf{O. Guziy(Huziy O.
    M.)}, Zheleznyak M.I., "Development Multi-Platform Version Of Decision
    Support System For Nuclear Emergency Management In Europe (Rodos) On Base Of Modern Java And Gis Technologies",
    Institute of Mathematical Machine and System Problems NAS of Ukraine, conference
    "Decision Support Systems. Theory and Practice", pp.121 - 124, Kiev, 2008.
     (\url{http://people.sca.uqam.ca/~huziy/pdf/ievdin_sppr2008.pdf})
\end{itemize}






% \section*{Hobbies and personal projects}
% \begin{itemize}
%   \item I have a personal \textbf{blog}, where I write some interesting things I
%   learn during my research activities: \\
%   \href{http://guziy.blogspot.com}{http://guziy.blogspot.com}.
%   \item Some of my \textbf{programming projects} are hosted at GitHub:
%   \href{https://github.com/guziy}{https://github.com/guziy}
%   \item I am interested in programming contests. For example, I participate in
%   the online \textbf{programming contest} at
%   \href{http://projecteuler.net}{http://projecteuler.net} using nickname guziy.
%   \item I like reading books (especially Stephen King's works), swimming and
%   I enjoy playing badminton.
% \end{itemize}



\end{document}
