% LaTeX Curriculum Vitae Template
%
% Copyright (C) 2004-2009 Jason Blevins <jrblevin@sdf.lonestar.org>
% http://jblevins.org/projects/cv-template/
%
% You may use use this document as a template to create your own CV
% and you may redistribute the source code freely. No attribution is
% required in any resulting documents. I do ask that you please leave
% this notice and the above URL in the source code if you choose to
% redistributemoti this file.

\documentclass[12pt,a4paper,sans]{moderncv}

\usepackage{datetime}

%----------------------------------------------------------------------------------------
%   NAME AND CONTACT INFORMATION SECTION
%----------------------------------------------------------------------------------------

\firstname{Oleksandr} % Your first name 
\familyname{Huziy} % Your last name

% All information in this block is optional, comment out any lines you don't need
\title{Curriculum Vitae}
\email{guziy.sasha@gmail.com}
\homepage{www.escer.uqam.ca/huziy_EN.html}{www.escer.uqam.ca/huziy\_EN.html}
\extrainfo{My blog: guziy.blogspot.com}
%\photo[70pt][0.4pt]{pictures/Huziy} 


\moderncvstyle{classic} % CV theme - options include: 'casual' (default),
% 'classic', 'oldstyle' and 'banking'
\moderncvcolor{blue} % CV color - options include: 'blue' (default), 'orange',
% 'green', 'red', 'purple', 'grey' and 'black'



\usepackage[scale=0.8]{geometry} % Reduce document margins
\setlength{\hintscolumnwidth}{4cm} % Uncomment to change the width of the dates column
%\setlength{\makecvtitlenamewidth}{10cm} % For the 'classic' style, uncomment to adjust the width of the space allocated to your name


\usepackage[utf8x]{inputenc} 


\begin{document}

\makecvtitle


\section{Education}


\cventry{05/2010--present}{Ph.D. in Earth and Atmospheric Sciences}{UQÀM
(Université du Québec à Montréal)}{}{}{}

\cventry{09/2009--04/2010}{M.Sc. in Atmospheric Sciences (classes only)}{UQÀM
(Université du Québec à Montréal)}{}{}{}

\cventry{09/2006--06/2008}{M.Sc. in applied physics and mathematics}{MIPT
(Moscow Institute of Physics and Technology, Kiev branch), department of
Control/Management and \newline Applied Mathematics.}{}{}{
Final dissertation: Development of weather prediction system for
Ukraine.}


\cventry{09/2002--06/2006}{B.Sc. in applied physics and mathematics}{MIPT
(Moscow Institute of Physics and Technology), department of
Problems of Physics and Power Engineering.}{}{}{Final dissertation: Modeling of transport of radioactive substances in the
atmosphere.}


\section*{Realisations and publications}


\begin{itemize}
    \item Jacinthe Clavet-Gaumont, Laxmi Sushama, Naveed Khaliq,
    \textbf{Oleksandr Huziy}, René Roy, "Canadian RCM projected changes to high
    flows for Québec watersheds using regional frequency analysis", submitted to International Journal of Climatology.
    
    \item \textbf{Oleksandr Huziy}, Laxmi Sushama, Naveed Khaliq, Bernhard
    Lehner, René Laprise, René Roy, "Analysis of Streamflow Characteristics over Northeastern Canada 
     in a Changing Climate", Climate Dynamics, 2012, DOI:
     10.1007/s00382-012-1406-0.
    \item \textbf{Oleksandr Huziy}, Laxmi Sushama, Naveed Khaliq, Bernhard Lehner,
     René Laprise, René Roy, "Analysis of Streamflow Characteristics over Northeastern Canada 
     in a Changing Climate", poster presentation at CMOS conference, Montreal,
     28 May - 1 June 2012.
    \item \textbf{Oleksandr Huziy}, Laxmi Sushama, Naveed Khaliq, Bernhard Lehner, René
    Laprise, René Roy: "Projected Changes to Streamflow Characteristics in
    North-East Canadian Basins as Simulated by the Canadian Regional Climate
    Model (CRCM4)", poster presentation at AGU Fall meeting, San Francisco,
    December 5, 2011.
    \item \textbf{Oleksandr (Sasha) Huziy}, Jacinthe Clavet-Gaumont, Laxmi
    Sushama, Naveed Khaliq, Bernhard Lehner, René Laprise, René Roy: "Projected changes to streamflow extremes in Quebec
    basins as simulated by the Canadian Regional Climate Model (CRCM4)",
    presentation at workshop at ESCER, UQÀM, Montreal (QC), May 25, 2011,
    (\url{http://people.sca.uqam.ca/~huziy/pdf/Workshop_UQAM_2011.pdf}).
   
    \item \textbf{Oleksandr Huziy}, Devis de recherche, "Development of a
    Lake-River Module for Canadian Regional Climate Model (CRCM5)", 2011,
    (\url{http://people.sca.uqam.ca/~huziy/pdf/devis_Huziy.pdf}).
   
    \item \textbf{Oleksandr Huziy}, "Hydrological Cycle in Current and Future Climate",
    projet réalisé dans le cadre du cours STA-9800: Dynamique du globe, 2011,\\
     (\url{http://people.sca.uqam.ca/~huziy/pdf/GlobeDynamicsReport.pdf}).
   
    \item \textbf{Oleksandr Huziy}, "Development of flow directions for use in
    distributed river models", projet réalisé dans le cadre du cours STA9850
    Concept de système en sciences de la Terre et de l'atmosphère, 2011,
    (\url{http://people.sca.uqam.ca/~huziy/pdf/flow_directions.pdf}).
   
    \item \textbf{Oleksandr Huziy}, "Atmospheric Boundary Layer over the Lake Surface",
           projet réalisé dans le cadre du cours "Couche limite atmosphérique",
           2010, (\url{http://people.sca.uqam.ca/~huziy/pdf/Report_Huziy_ABL.pdf}).
   
    \item \textbf{Oleksandr Huziy}, "Modélisation d’un nuage chaud stratiforme",
       projet réalisé dans le cadre du cours "Convection et precipitation", 2010,
       (\url{http://people.sca.uqam.ca/~huziy/pdf/Huziy_SCA7050_rapport.pdf}).

    \item Oleksandr Huziy et André Monette, "Identification et Prévision du
    Brouillard en Utilisant les Données Satellitaires", projet réalisé dans le cadre du cours Météorologie et télédetection,
       2010, (\url{http://people.sca.uqam.ca/~huziy/pdf/Report_Monette_HUziy_teledetection.pdf}).

    \item \textbf{Guziy A.M. (Huziy O. M.)}, Kovalets I.V., Kushchan A.A.,
    Zheleznyak M.J.
    "Statistical downscaling of results of the forecasting system MM5–Ukraine using neural net technology.",
    workshop presentation, Saint-Petersburg, 2008, (\url{http://people.sca.uqam.ca/~huziy/pdf/presentation_spb2008.pdf}).

    \item I. Ievdin, D. Treebushny, I. Kovalets, \textbf{O. Guziy(Huziy O. M.)}, M.I.
    Zheleznyak, "Development Multi-Platform Version Of Decision Support System For Nuclear Emergency Management In Europe (Rodos)
    On Base Of Modern Java And Gis Technologies", 
    Institute of Mathematical Machine and System Problems NAS of Ukraine, conference
    "Decision Support Systems. Theory and Practice", pp.121 - 124, Kiev, 2008.
     (\url{http://people.sca.uqam.ca/~huziy/pdf/ievdin_sppr2008.pdf})

\end{itemize}

\section*{Experience}
\begin{itemize}
  
  \item (02/2012 - 03/2012, 05/2012):
  \textbf{research assistant} at centre ESCER (l'étude et simulation du climat à
  l'échelle régionale), Université du Québec à Montréal.\\
  Tasks:
  \begin{enumerate}
    \item Analysis of output from regional climate models.
    \item Support with GIS applications.
  \end{enumerate}

  \item (08/2008 - 10/2008): 
   \textbf{junior researcher} at Institute of Mathematical
  Machines and Problems, Kiev, Ukraine. \\
  Tasks:
  \begin{enumerate}
    \item Scientific support of a weather forecasting system at
    \href{http://meteoprog.com}{http://meteoprog.com}.
    \item Development of a decision support system for nuclear emergency
    management in Europe (Rodos).
  \end{enumerate}
\end{itemize}

\section*{Relevant Skills}
\begin{itemize}  
  \item Programming languages in which I have written programs: \textbf{ Python,
  Java, MATLAB, C/C++, Fortran, R, Linux shell scripting}. 
  \item I used parallel models WRF (Weather Research and Forecasting) and CRCM
  (Canadian Regional Climate Model), and wrote parallel programs in Fortran
  (MPI), Python (multithreading and multiprocessing) and Java (threads).
\end{itemize}


\section*{Scholarships}

\begin{itemize}
    \item (2009-2012) Fee waiver scholarship for international students given
    by gouvernment of Quebec.
    \item (2011): excellence scholarship from Hydro-Quebec.
    \item (2010): excellence scholarship from Hydro-Quebec.
    \item (2009): excellence scholarship FARE.
\end{itemize}


% \section*{Hobbies and personal projects}
% \begin{itemize}
%   \item I have a personal \textbf{blog}, where I write some interesting things I
%   learn during my research activities: \\
%   \href{http://guziy.blogspot.com}{http://guziy.blogspot.com}.
%   \item Some of my \textbf{programming projects} are hosted at GitHub:
%   \href{https://github.com/guziy}{https://github.com/guziy}
%   \item I am interested in programming contests. For example, I participate in
%   the online \textbf{programming contest} at
%   \href{http://projecteuler.net}{http://projecteuler.net} using nickname guziy.
%   \item I like reading books (especially Stephen King's works), swimming and
%   I enjoy playing badminton.
% \end{itemize}



\end{document}

