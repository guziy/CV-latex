% LaTeX Curriculum Vitae Template
%
% Copyright (C) 2004-2009 Jason Blevins <jrblevin@sdf.lonestar.org>
% http://jblevins.org/projects/cv-template/
%
% You may use use this document as a template to create your own CV
% and you may redistribute the source code freely. No attribution is
% required in any resulting documents. I do ask that you please leave
% this notice and the above URL in the source code if you choose to
% redistributemoti this file.

%%%%%%%%%%%%%%%%%%%%%%%%%%%%%%%%%%%%%%%%%
% "ModernCV" CV and Cover Letter
% LaTeX Template
% Version 1.1 (9/12/12)
%
% This template has been downloaded from:
% http://www.LaTeXTemplates.com
%
% Original author:
% Xavier Danaux (xdanaux@gmail.com)
%
% License:
% CC BY-NC-SA 3.0 (http://creativecommons.org/licenses/by-nc-sa/3.0/)
%
% Important note:
% This template requires the moderncv.cls and .sty files to be in the same 
% directory as this .tex file. These files provide the resume style and themes 
% used for structuring the document.
%
%%%%%%%%%%%%%%%%%%%%%%%%%%%%%%%%%%%%%%%%%

%----------------------------------------------------------------------------------------
%   PACKAGES AND OTHER DOCUMENT CONFIGURATIONS
%----------------------------------------------------------------------------------------

\documentclass[12pt,a4paper,sans]{moderncv} % Font sizes: 10, 11, or 12; paper sizes: a4paper, letterpaper, a5paper, legalpaper, executivepaper or landscape; font families: sans or roman

\moderncvstyle{classic} % CV theme - options include: 'casual' (default),
% 'classic', 'oldstyle' and 'banking'
\moderncvcolor{blue} % CV color - options include: 'blue' (default), 'orange',
% 'green', 'red', 'purple', 'grey' and 'black'



\usepackage[scale=0.8]{geometry} % Reduce document margins
\setlength{\hintscolumnwidth}{4cm} % Uncomment to change the width of the dates column
%\setlength{\makecvtitlenamewidth}{10cm} % For the 'classic' style, uncomment to adjust the width of the space allocated to your name


\usepackage[utf8x]{inputenc} 


%----------------------------------------------------------------------------------------
%   NAME AND CONTACT INFORMATION SECTION
%----------------------------------------------------------------------------------------

\firstname{Oleksandr} % Your first name 
\familyname{Huziy} % Your last name

% All information in this block is optional, comment out any lines you don't need
\title{Curriculum Vitae}
\email{guziy.sasha@gmail.com}
\homepage{www.escer.uqam.ca/huziy_EN.html}{www.escer.uqam.ca/huziy\_EN.html}
\extrainfo{My blog: guziy.blogspot.com}
%\photo[70pt][0.4pt]{pictures/Huziy} 


\begin{document}

% Comment the following lines to use the default Computer Modern font
% instead of the Palatino font provided by the mathpazo package.
% Remove the 'osf' bit if you don't like the old style figures.
%\usepackage[T1]{fontenc}
%\usepackage[sc]{mathpazo}
\makecvtitle % Print the CV title


\section{Éducation}
\cventry{05/2010--présent}{Doctorat en sciences de la Terre et de l'atmosphère}{UQÀM (Université du Québec à Montréal)}{}{}{}
\cventry{09/2009--04/2010}{Maîtrise en sciences de l'atmosphère (les cours seulement)}{UQÀM (Université du Québec à Montréal)}{}{}{}
\cventry{09/2006--06/2008}{Maîtrise en physique et mathématiques appliquées, département de contrôle/gestion et de mathématiques appliquées}{MIPT (Institut de Moscou de physique et de la technologie, département à Kiev), département de contrôle/gestion et de mathématiques appliquées.}{}{}{Projet final: Développement du système de prévision du temps pour l'Ukraine.}
\cventry{09/2002--06/2006}{Baccalauréat en physique et mathématiques appliquées, département des Problèmes de physique et d'énergétique}{MIPT (Institut de Moscou de physique et de la technologie), département
des problèmes de physique et génie énergétique.}{}{}{Projet final: Modélisation du transport des substances radioactives dans
l'atmosphère.}



\section{Projet de doctorat}
\cvitem{Titre}{Développement d'un module lac-rivière pour le Modèle Régional
Canadien du Climat (MRCC5)}
\cvitem{Directeur}{Laxmi Sushama}
\cvitem{Co-directeur}{René Laprise}


\section{Expérience}
%%%%%%%%%%%%%%%%%%%%%%%% CNRCWP coordinator
\cvitem{07/2013--présent}{\textbf{Coordinateur} du Réseau CNRCWP (Canadian
Network for Regional Climate and Weather Processes)}
\cvitem{}{Fonctions:}
\cvitem{}{\labelitemii Maintenance et mis à jour du site internet du Réseau
\url{cnrcwp.uqam.ca}.}
\cvitem{}{\labelitemii Documentation, organisation des réunions et conférences.}

%%%%%%%%%%%%%%%%%%%%%%%%% TA at UQAM
\cvitem{10/2012--12/2012}{
  \textbf{Auxiliaire d'enseignement (démonstrateur)}, cours: Hydrologie,
  Université du Québec à Montréal.} 
  
\cvitem{}{Fonctions:}
\cvitem{}{\labelitemii Animation et correction des travaux pratiques.}
\cvitem{}{\labelitemii Support technique.}


%%%%%%%%%%%%%%%%%%%%%%%% RA at UQAM
\cvitem{02/2012--03/2012, 05/2012}{
  \textbf{Assistant de recherche} centre ESCER (l'étude et simulation du
  climat à l'échelle régionale), Université du Québec à Montréal.}
\cvitem{}{Fonctions:}
\cvitem{}{\labelitemii Analyse des sorties des modèles régionaux du climat.}
\cvitem{}{\labelitemii Support avec applications SIG.}


%%%%%%%%%%%%%%%%%%%%%%%% RS at IMMSP
\cvitem{08/2008--10/2008}{ 
   \textbf{Chercheur junior} à l'Institut de problèmes des machines et des
   systèmes de mathématique, Kiev, Ukraine.}
\cvitem{}{Fonctions:}
\cvitem{}{\labelitemii Support scientifique du système de prévision du temps
    \url{http://meteoprog.com}.}
\cvitem{}{\labelitemii Développement du système de support des décisions pour la
    gestion des situations d'urgence nucléaire en Europe (RODOS).}





\section{Réalisations et publications}

\vspace{0.5cm}
\subsection{Conférences et séminaires}
\vspace{0.5cm}
\begin{itemize}
  
   \item \textbf{Huziy, O.}, Sushama L., Khaliq M.N., Laprise R., Lehner B. and
   Roy R., 2013. Analysis of streamflow characteristics over northeastern
   Canada in a changing climate. Canadian Meteorological and Oceanographic Society (CMOS), Saskatoon.
   
   \item \textbf{Huziy, O.}, Sushama L. and Laprise R., 2013. Impact of lakes
   and rivers on the regional climate and hydrology of northeast Canada. Workshop on Regional Climate modelling and Diagnostics, UQAM, Montreal.
  
  
    \item \textbf{Huziy O.}, Sushama L., Khaliq M.N., Lehner B., Laprise R., Roy
    R., "Analysis of Streamflow Characteristics over Northeastern Canada in a Changing Climate", poster presentation at CMOS conference, Montreal,
     28 May - 1 June 2012.
     
    \item \textbf{Huziy O.}, Sushama L., Khaliq M.N., Lehner B., Laprise R.,
    Roy R.: "Projected Changes to Streamflow Characteristics in North-East
    Canadian Basins as Simulated by the Canadian Regional Climate Model (CRCM4)", poster presentation at AGU Fall meeting, San Francisco,
    December 5, 2011.
    
    \item \textbf{Huziy O.}, Clavet-Gaumont J., Sushama L., Khaliq M.N.,
    Lehner B., Laprise R., Roy R.: "Projected changes to streamflow extremes
    in Quebec basins as simulated by the Canadian Regional Climate Model (CRCM4)", presentation at workshop at ESCER, UQÀM, Montreal (QC), May 25, 2011,
    (\url{http://people.sca.uqam.ca/~huziy/pdf/Workshop_UQAM_2011.pdf}). 
\end{itemize}


\vspace{0.5cm}
\subsection{Publications et rapports}
\vspace{0.5cm}
\begin{itemize}
  
    \item Jacinthe Clavet-Gaumont, Sushama L., Khaliq M.N.,
    \textbf{Huziy O.}, Roy R., "Canadian RCM projected changes to high
    flows for Québec watersheds using regional frequency analysis", International Journal of Climatology.
    
    \item \textbf{Huziy O.}, Sushama L., Khaliq M.N., Bernhard
    Lehner, René Laprise, Roy R., "Analysis of Streamflow Characteristics over Northeastern Canada 
     in a Changing Climate", Climate Dynamics, 2012, DOI:
     10.1007/s00382-012-1406-0.

   
    \item \textbf{Huziy O.}, Devis de recherche, "Development of a
    Lake-River Module for Canadian Regional Climate Model (CRCM5)", 2011,
    (\url{http://people.sca.uqam.ca/~huziy/pdf/devis_Huziy.pdf}).
   
    \item \textbf{Huziy O.}, "Hydrological Cycle in Current and Future Climate",
    projet réalisé dans le cadre du cours STA-9800: Dynamique du globe, 2011,\\
     (\url{http://people.sca.uqam.ca/~huziy/pdf/GlobeDynamicsReport.pdf}).
   
    \item \textbf{Huziy O.}, "Development of flow directions for use in
    distributed river models", projet réalisé dans le cadre du cours STA9850
    Concept de système en sciences de la Terre et de l'atmosphère, 2011,
    (\url{http://people.sca.uqam.ca/~huziy/pdf/flow_directions.pdf}).
   
    \item \textbf{Huziy O.}, "Atmospheric Boundary Layer over the Lake Surface",
           projet réalisé dans le cadre du cours "Couche limite atmosphérique",
           2010, (\url{http://people.sca.uqam.ca/~huziy/pdf/Report_Huziy_ABL.pdf}).
   
    \item \textbf{Huziy O.}, "Modélisation d’un nuage chaud stratiforme",
       projet réalisé dans le cadre du cours "Convection et precipitation", 2010,
       (\url{http://people.sca.uqam.ca/~huziy/pdf/Huziy_SCA7050_rapport.pdf}).

    \item Huziy O. et Monette A., "Identification et Prévision du
    Brouillard en Utilisant les Données Satellitaires", projet réalisé dans le cadre du cours Météorologie et télédetection,
       2010, (\url{http://people.sca.uqam.ca/~huziy/pdf/Report_Monette_HUziy_teledetection.pdf}).

    \item \textbf{Guziy A.M. (Huziy O. M.)}, Kovalets I.V., Kushchan A.A.,
    Zheleznyak M.J.
    "Statistical downscaling of results of the forecasting system MM5–Ukraine using neural net technology.",
    workshop presentation, Saint-Petersburg, 2008, (\url{http://people.sca.uqam.ca/~huziy/pdf/presentation_spb2008.pdf}).

    \item Ievdin I., Treebushny D., Kovalets I., \textbf{O. Guziy(Huziy O.
    M.)}, Zheleznyak M.I., "Development Multi-Platform Version Of Decision
    Support System For Nuclear Emergency Management In Europe (Rodos) On Base Of Modern Java And Gis Technologies", 
    Institute of Mathematical Machine and System Problems NAS of Ukraine, conference
    "Decision Support Systems. Theory and Practice", pp.121 - 124, Kiev, 2008.
     (\url{http://people.sca.uqam.ca/~huziy/pdf/ievdin_sppr2008.pdf})

\end{itemize}


\section{Bourses et distinctions}
    \cvitem{2009--2012}{ Bourses d'exemptions des droits
    de scolarité supplémentaires pour les étudiants étrangers données par le
    gouvernement du Québec.}
    \cvitem{2012}{Bourse d'éxcellence d'Hydro-Québec.} 
    \cvitem{2011}{Bourse d'éxcellence d'Hydro-Québec.} 
    \cvitem{2010}{Bourse d'éxcellence d'Hydro-Québec.}
    \cvitem{2009}{Bourse d'éxcellence FARE (fonds à l'accessibilité et à la
    réussite des études). }


\section{Services academiques}
Critique pour International Journal of Climate.


\end{document}

